%%%%%%%%%%%%%%%%%%%%
% INTRODUCCION
%%%%%%%%%%%%%%%%%%%%

\noindent Este trabajo se enmarca dentro del proyecto ``Foodprice'', una iniciativa de la Pontificia Universidad Javeriana Cali, el cual busca desarrollar un paquete en el lenguaje R para el c\'alculo de dietas de costo m\'inimo que garanticen una nutrici\'on adecuada seg\'un las caracter\'isticas de cada persona. Se seleccion\'o el trabajo ``Sistema accesible para la generaci\'on de dietas saludables de costo m\'inimo con consideraciones de econom\'ia y nutrici\'on personalizadas'' como una forma de expandir el alcance del proyecto Foodprice, maximizando as\'i su impacto en la sociedad.
\\
\\
\noindent La mala alimentaci\'on en Colombia representa una problem\'atica grave que afecta, en especial, a las familias de escasos recursos \cite{sarmientoDesnutricionColombiaDesde}. Esta situaci\'on obedece a diversos factores, entre ellos la falta de educaci\'on alimentaria y nutricional, el elevado costo de los alimentos y las restricciones presupuestarias a las que estas familias se enfrentan.
\\
\\
\noindent Tanto el gobierno como diversas organizaciones sin \'animo de lucro han realizado esfuerzos para reducir los problemas de desnutrici\'on \cite{HambreCeroAgenda}. Sin embargo, estos esfuerzos no han sido suficientes, ya que la problem\'atica parece haberse agudizado en los \'ultimos a\~{n}os \cite{aumentoDesnutricion}. Los programas de alimentaci\'on dirigidos a comunidades vulnerables, as\'i como las campa\~{n}as de educaci\'on nutricional, buscan contribuir a la soluci\'on de este problema, aunque a\'un queda mucho trabajo por hacer.
\\
\\
\noindent En respuesta a esta situaci\'on, se plantea el desarrollo de este proyecto, cuyo objetivo es crear un sistema que facilite el uso del software desarrollado en el marco del proyecto Foodprice, permitiendo la generaci\'on de dietas de costo m\'inimo que cubran las necesidades nutricionales b\'asicas del usuario, considerando sus caracter\'isticas espec\'ificas. Una parte fundamental del proyecto es el desarrollo de una aplicaci\'on web, con el fin de hacerlo m\'as accesible y llegar a un mayor n\'umero de personas, al no requerir conocimientos t\'ecnicos para su uso. Con ello, se espera aportar a la reducci\'on de los \'indices de desnutrici\'on en poblaciones de bajos recursos, proporcionando una herramienta que permita identificar qu\'e ingredientes son necesarios para una dieta lo m\'as econ\'omica posible, incentivando as\'i una mejor alimentaci\'on a pesar de las dificultades econ\'omicas que puedan presentarse.
