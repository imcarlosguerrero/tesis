%%%% Conclusiones

\noindent Con los resultados obtenido en este proyecto, se evidencia que si es posible el desarrollo de un sistema accesible para la generaci\'on de dietas saludables de costo m\'inimo con consideraciones de econom\'ia y nutrici\'on personalizadas.
\\
\\
Sin embargo, a pesar de haber cumplido el objetivo principal de este trabajo de grado, se destaca como el contar con una base m\'ejor estructurada hubiera facilitado el desarrollo, debido a que muchas de las complicaciones que se tuvo durante la elaboraci\'on fueron por malas pr\'acticas de programaci\'on y de estructuraci\'on que se encontraban en el proyecto base; a pesar de esto, se logr\'o crear una plataforma que cuenta con una interfaz con buena accesibilidad, y que permite visualizar de manera gr\'afica y con precios realistas una dieta que cumpla con las diferentes necesidades de las personas.
\\
\\
El sistema permite la creaci\'on desde dietas de subsistencia hasta dietas enteramente saludables y con variedad de alimentos, donde inclusive se permite la eliminaci\'on de productos que puedan no ser del agrado del usuario o que de plano no puedan consumir, ya sea por una afecci\'on m\'edica, alergias o por simple disgusto hacia un alimento en particular, con lo que la construcci\'on de una dieta saludable y personalizada es algo completamente viable con el sistema desarrollado.
\\
\\
Sin embargo, aun quedan dilemas por resolver para que el sistema pueda funcionar a toda su capacidad, no se logr\'o que la base de datos se actualice de forma constante, debido al gasto computacional que esto requiere por el proceso de web scraping, por lo que el proyecto todav\'ia tiene mucho potencial de crecer y mejorar.