\section{Marco de Referencia}
\subsection{\'Areas Tem\'aticas}
% Listar las \'areas tem\'aticas del proyecto. Para las \'areas propias de la disciplina, utilice las categor\'ias de la ACM. Por ejemplo: D.1.3: Software - Programming Techniques -Concurrent Programming. Ver clasificaci\'on en  (\url{http://www.acm.org/about/class/ccs98-html}). 

\noindent De acuerdo con el sistema de clasificaci\'on computacional ACM, las \'areas tem\'aticas que abarca el proyecto son:
\begin{itemize}
    \item  Applied Computing - Life and Medical Sciences - Health Informatics
    \item Information Systems - Decision Support Systems
    \item Human-Centered Computing - Accessibility
\end{itemize}

\subsection{Marco Te\'orico}

\noindent La malnutrici\'on en Colombia es un fen\'omeno que afecta directamente a las personas m\'as pobres de nuestro pa\'is, en el marco de este proyecto se utilizar\'a la computaci\'on para proveer de dietas saludables de costo m\'inimo con consideraciones de econom\'ia y nutrici\'on personalizada, basada en ciertos par\'ametros espec\'ificos. Ser\'a necesario definir algunos conceptos claves en el proyecto. Entre los cuales se encuentran: nutrici\'on, dieta saludable, poblaci\'on econ\'omicamente vulnerable, aplicaci\'on web.


\subsubsection{Nutrici\'on}

\noindent La nutrici\'on es un aspecto cr\'itico de la salud y el desarrollo. La buena nutrici\'on guarda relaci\'on con la buena salud del lactante, el ni\~{n}o y la madre; sistemas inmunitarios m\'as fuertes; embarazos y partos m\'as seguros; menos riesgos de enfermedades no transmisibles (tales como diabetes y enfermedades cardiovasculares) y longevidad.\cite{Nutricion}


\subsubsection{Dieta Saludable} \label{DietaSaludable}
\noindent Una dieta saludable es una de las bases para la salud, el bienestar, el crecimiento \'optimo y el desarrollo, y protege contra todas las formas de malnutrici\'on. Una dieta malsana es uno de los principales riesgos para la carga mundial de morbilidad, principalmente en lo que se refiere a enfermedades no transmisibles como las enfermedades cardiovasculares, la diabetes y el c\'ancer.\cite{DietaSana}



\subsubsection{Poblaci\'on Econ\'omicamente Vulnerable}
\noindent  La l\'inea de pobreza monetaria es el valor en dinero que necesita una persona al mes para adquirir una canasta b\'asica de alimentos, servicios y otros bienes m\'inimos para vivir. Si una persona tiene un ingreso menor a este valor se considera en situaci\'on de pobreza monetaria. Por otra parte, la l\'inea de pobreza monetaria extrema es el valor en dinero que necesita una persona mensualmente para adquirir una canasta b\'asica alimentaria que le provea el m\'inimo requerimiento cal\'orico para subsistir.\cite{PobrezaMonetaria}

\subsubsection{Aplicaci\'on Web}
\noindent  Una aplicaci\'on web es un software que se ejecuta en su navegador web. Las empresas tienen que intercambiar informaci\'on y prestar servicios de forma remota. Utilizan aplicaciones web para conectarse con los clientes de forma c\'omoda y segura. Las aplicaciones web permiten a los usuarios acceder a funciones complejas sin instalar ni configurar software.\cite{WhatWebApp}



\section{Trabajos relacionados}\label{trabajosRelacionados}

En esta secci\'on se presentan diferentes trabajos de investigaci\'on, los cuales se basan en tem\'atocas similares a las tratadas en este proyecto, o usan tecnologias relacionadas.

\subsection{Diet recommendation system using machine learning}
En este trabajo, se habla sobre la utilizaci\'on de aprendizaje de m\'aquina para generar dietas personalizadas basadas en las caracter\'isticas de la persona, as\'i como sus objetivos. Se utilizan m\'etodos conocidos de aprendizaje de m\'aquina como son random forest y LSTM, para dar recomendaciones de dietas tomando en cuenta sus caracter\'isticas f\'isicas para calcular su BMI y determinar si la persona tiene sobrepeso, est\'a debajo de su peso sano, o es saludable; y en base a esto, ofrece 3 tipos de dietas, para bajar de peso, mantenerse, o aumentar. Este trabajo se relaciona con el propuesto en este proyecto, ya que en ambos se utilizar\'a el avance tecnol\'ogico y diferentes estrategias matem\'aticas para determinar dietas en base a la caracter\'istica de la persona, y lo que necesita para encontrarse en un estado m\'as saludable.\cite{DietRecommendation}


\subsection{Food budget standards and dietary adequacy in low-income families}
Este trabajo trata sobre el presupuesto alimentario y la adecuaci\'on diet\'etica en familias de bajos recursos, el cual consiste en a trav\'es de diversas encuestas y an\'alisis, estudiar los h\'abitos de gastos alimenticios en familias inglesas que cumplen con caracter\'isticas econ\'omicas parecidas respecto a sus ingresos. En este trabajo se observa como hay una relaci\'on entre tanto caracter\'isticas f\'isicas (como sexo y edad), y caracter\'isticas sociales (como estado civil), con el presupuesto que destinan a la alimentaci\'on, y, adem\'as, como esto influye directamente en su adecuada alimentaci\'on y posible desarrollo en los ni\~{n}os. Esto se relaciona con el proyecto propuesto debido a que el presupuesto destinado para la alimentaci\'on por parte de familias de bajos recursos, es uno de los factores claves a tomar en cuenta para la generaci\'on adecuada de dietas que se adapten a sus necesidades.\cite{leeDietPlanningMachine2021}


\subsection{Diet Planning with machine learning}
En este trabajo se trata el problema de planear dietas para ni\~{n}os utilizando aprendizaje de m\'aquina para planear la alimentaci\'on de centros de cuidados de ni\~{n}os de Corea del Sur. Se menciona sobre c\'omo el aprendizaje de m\'aquina toma en cuenta factores que con modelos matem\'aticos son complicadas de tomar en cuenta, como lo son la naturalidad y el disfruta de la combinaci\'on de comidas, as\'i como los valores nutricionales que estas aportan para lograr una dieta sana y que no sea inc\'omoda de seguir. Esto se relaciona con el trabajo del presente proyecto, ya que la capacidad de combinar comidas de formas en que la dieta suministrada no sea poco apetitosa o antinatural, son factores que se tendr\'an presentes tambi\'en en este proyecto, adem\'as de poder cumplir con los valores nutricionales necesarios para una vida sana.\cite{nelsonFoodBudgetStandards2002}


